% Options for packages loaded elsewhere
\PassOptionsToPackage{unicode}{hyperref}
\PassOptionsToPackage{hyphens}{url}
%
\documentclass[
]{article}
\usepackage{amsmath,amssymb}
\usepackage{lmodern}
\usepackage{ifxetex,ifluatex}
\ifnum 0\ifxetex 1\fi\ifluatex 1\fi=0 % if pdftex
  \usepackage[T1]{fontenc}
  \usepackage[utf8]{inputenc}
  \usepackage{textcomp} % provide euro and other symbols
\else % if luatex or xetex
  \usepackage{unicode-math}
  \defaultfontfeatures{Scale=MatchLowercase}
  \defaultfontfeatures[\rmfamily]{Ligatures=TeX,Scale=1}
\fi
% Use upquote if available, for straight quotes in verbatim environments
\IfFileExists{upquote.sty}{\usepackage{upquote}}{}
\IfFileExists{microtype.sty}{% use microtype if available
  \usepackage[]{microtype}
  \UseMicrotypeSet[protrusion]{basicmath} % disable protrusion for tt fonts
}{}
\makeatletter
\@ifundefined{KOMAClassName}{% if non-KOMA class
  \IfFileExists{parskip.sty}{%
    \usepackage{parskip}
  }{% else
    \setlength{\parindent}{0pt}
    \setlength{\parskip}{6pt plus 2pt minus 1pt}}
}{% if KOMA class
  \KOMAoptions{parskip=half}}
\makeatother
\usepackage{xcolor}
\IfFileExists{xurl.sty}{\usepackage{xurl}}{} % add URL line breaks if available
\IfFileExists{bookmark.sty}{\usepackage{bookmark}}{\usepackage{hyperref}}
\hypersetup{
  pdftitle={Riesgos\_Competitivos},
  hidelinks,
  pdfcreator={LaTeX via pandoc}}
\urlstyle{same} % disable monospaced font for URLs
\usepackage[margin=1in]{geometry}
\usepackage{graphicx}
\makeatletter
\def\maxwidth{\ifdim\Gin@nat@width>\linewidth\linewidth\else\Gin@nat@width\fi}
\def\maxheight{\ifdim\Gin@nat@height>\textheight\textheight\else\Gin@nat@height\fi}
\makeatother
% Scale images if necessary, so that they will not overflow the page
% margins by default, and it is still possible to overwrite the defaults
% using explicit options in \includegraphics[width, height, ...]{}
\setkeys{Gin}{width=\maxwidth,height=\maxheight,keepaspectratio}
% Set default figure placement to htbp
\makeatletter
\def\fps@figure{htbp}
\makeatother
\setlength{\emergencystretch}{3em} % prevent overfull lines
\providecommand{\tightlist}{%
  \setlength{\itemsep}{0pt}\setlength{\parskip}{0pt}}
\setcounter{secnumdepth}{-\maxdimen} % remove section numbering
\ifluatex
  \usepackage{selnolig}  % disable illegal ligatures
\fi

\title{Riesgos\_Competitivos}
\author{}
\date{\vspace{-2.5em}}

\begin{document}
\maketitle

\{r setup, include=FALSE\} knitr::opts\_chunk\$set(echo = TRUE)

\hypertarget{introducciuxf3n-y-preeliminares-en-el-anuxe1lisis-de-datos-de-un-hospital-impacto-del-estado-del-estado-de-neumonuxeda-al-momento-de-la-admisiuxf3n-en-la-unidad-de-terapia-intensiva}{%
\section{Introducción y preeliminares en el análisis de datos de un
hospital: Impacto del estado del estado de neumonía al momento de la
admisión en la unidad de terapia
intensiva}\label{introducciuxf3n-y-preeliminares-en-el-anuxe1lisis-de-datos-de-un-hospital-impacto-del-estado-del-estado-de-neumonuxeda-al-momento-de-la-admisiuxf3n-en-la-unidad-de-terapia-intensiva}}

En este trabajo, abordamos el análisis de supervivencia de riesgos
competitivos, por lo que el esquema de trabajo cambia un poco respecto
al visto en clase, que es realizar el modelo de Cox. Así mismo, como
tenemos otro enfoque (Al tratar ahora desde el punto de vista de riesgos
competitivos), no aplican las estimaciones de Kaplan-Meier, sino otros
estimadores.

Trabajaremos con la base sir.adm; dispinible en el paquete ``mvna'',
llamado así por las iniciales de ``Multivariate Nelson-Aalen
estimator'', que permite estimar de manera no paramétrica los riesgos
acumulados de transición de modelos de Markov multi-estados arbitrarios,
usando dicho estimador.

\{r\} library(mvna) \#Estimadores de Nelson-Aalen library(lattice)
\#Apoyo gráfico library(cmprsk) \#Paquetería enfocada a análisis de
subdistribución de riesgos competitivos library(etm) \#Paquetería que
nos permite estimar la matriz de transiciones de probabilidad para
modelos multiestado de tiempo no homogéneo de espacio finito, usando el
estimador Aalen-Johansen

Cargamos la base de datos \{r\} data(sir.adm)

Veamos las primeras 5 observaciones \{r\} head(sir.adm)

El datasetcontiene una submuestra aleatoria de 747 pacientes y 6
variables:

id: Es un id generado aleatoriamente para cada paciente

pneu: Es un indicador de si el paciente presentaba neumonía (1) o no la
presentaba (0) al momento de ser admitido en el estudio

status: Un indicador del estatus de la observación: 0 es una observación
censurada, 1 es que el paciente fue dado de alta, 2 es que el paciente
murió

time: Es el tiempo en días

age: La edad del paciente cuando entró al estudio

sex: F para mujer, M para hombre.

\{r\} sum(is.na.data.frame(sir.adm))

Vemos que no hay ningún dato faltante (NA) en el dataset.

Estas 747 pacientes son de SIR 3 (Spread of nosocomial Infections and
Resistant pathogens, es decir, Esparcimiento de infecciones nosocomiales
y patógenos resistentes), un estudio conjunto en el hospital Charité de
la Universidad de Berlin, Alemania, con una valoración prospectiva de
datos para examinar el efecto de infecciones adquiridad en el hospital,
en terapia-intesiva.

Notemos que se presenta una censura por la derecha.

El dataset contiene información en el estado de admisión de neumonía,
tiempo de estadía en la unidad de terapia intensiva y ``desenlace o
resultado del tratamiento en unidad intensiva'', es decir, si se le dio
de alta al paciente o este murió.

La neumonía es una infección severa, de la cual se tiene la sospecha,
causa el requerimiento de cuidados adicionales (Es decir, una prolongada
estadía en unidades de cuidado intensivo) y de incrementar la
mortalidad.

\{r\} sum(sir.adm\$status==0)

14 Observaciones censuradas, es decir, que seguían en unidad intensiva
al final del estudio.

\{r\} sum(sir.adm\$status==1)

657 Pacientes que fueron dados de alta

\{r\} sum(sir.adm\$status==2)

76 pacientes fallecieron

\{r\} sum(sir.adm\(status==2 & sir.adm\)pneu==1)

21 de los pacientes que murieron tenían neumonía al momento de ser
admitidos. \{r\} sum(sir.adm\(status==2 & sir.adm\)pneu==0)

55 de los pacientes que murieron, no tenían neumonía al momento de ser
admitidos

Esta base la elegimos porque nos parece un buen ejemplo de riesgos
competitvos por lo siguiente:

Se investiga el tiempo hasta el final de la estadía y el estatus de el
estado final: Si se dio de alta o murió en el hospital. Un desafío en el
análisis de este dataset es que se ha encontrado que la neumonía
incrementa la probabilidad de morir en el hospital, pero parece no tener
efecto en el riesgo de muerte, es decir, no tiene efecto en la
probabilidad diaria de morir en el hospital, dado que uno sigue vivo y
en la unidad de cuidados intensivos al inicio del día.

Los puntos finales competitivos son el ser dado de alta y morir en la
unidad de terapia intensiva

\hypertarget{anuxe1lisis-descriptivo-y-enfoque-no-paramuxe9trico}{%
\section{Análisis descriptivo y enfoque no
paramétrico}\label{anuxe1lisis-descriptivo-y-enfoque-no-paramuxe9trico}}

\hypertarget{funciuxf3n-de-riesgos-acumulados}{%
\subsection{Función de riesgos
acumulados}\label{funciuxf3n-de-riesgos-acumulados}}

Necesitamos modificar primero el dataset en un dataset de tipo
multiestado. Además, renombramos los datos para que, el evento de
interés que es la muerte, corresponda al estado 1, y el estado 2 el
evento competitivo: \{r\} \#Transformamos sir.adm a un dataset de tipo
multiestado to \textless-
ifelse(sir.adm\(status == 0, "cens", ifelse(sir.adm\)status == 1, 2, 1))
my.sir.data \textless- data.frame(id =
sir.adm\(id, from = 0, to, time = sir.adm\)time, pneu = sir.adm\$pneu)

Notemos que my.sir.data tiene un componente pneu con el estatus de
neumonía al momento de admisión. Revisamos que hicimos bien el
remombramiento: \{r\} table(my.sir.data\$to)

Lo hicimos de manera correcta.

Antes de continuar, necesitamos describir el modelo de riesgos
competitivos muliestado siguiente:

\{r, echo=FALSE, out.width = `100\%'\}
knitr::include\_graphics(``Imagen1.PNG'')

\{r, echo=FALSE, out.width = `100\%'\}
knitr::include\_graphics(``Imagen\_2.PNG'')

Lo hacemos definiendo una matriz de valores lógicos inficando los
posibles tipos de transición en nuestro modelo multiestado:

\{r\} tra \textless- matrix(FALSE, ncol = 3, nrow = 3) dimnames(tra)
\textless- list(c(``0'', ``1'', ``2''), c(``0'', ``1'', ``2''))\\
tra{[}1, 2:3{]} \textless- TRUE tra

Esta matriz nos dice que un individup se puede mover del estado 0 al
estado 1, y del estado 0 al estado 2, pero las transiciones en el
sentido contrario no son posibles; además los valores en diagonal están
como falsos: Las transiciones de un estado a sí mismo no están
modeladas. No es necesario un modelo para dicha ``transición''. Los
individuos que no hacen una transición a uno de los dos pares
competitivos al tiempo t, permanece en el estado inicial 0 a tiempo t.

Ahora, calculamos la función de acumulación de riesgos específicos para
muerte y dados de alta, respectivamente, y estratificados por su estatus
de neumonía al momento de admisión:

\{r\} \#\# sin neumonía my.nelaal.nop \textless-
mvna(my.sir.data{[}my.sir.data\$pneu == 0, {]}, c(``0'', ``1'', ``2''),
tra, ``cens'') \#\# con neumonía my.nelaal.p \textless-
mvna(my.sir.data{[}my.sir.data\$pneu == 1, {]}, c(``0'', ``1'', ``2''),
tra, ``cens'')

Graficamos:

\{r\} \# Estilo de la gráfica ltheme \textless- canonical.theme(color =
FALSE) ltheme\(strip.background\)col \textless- ``white''
lattice.options(default.theme = ltheme)

\{r\} \#Plot para los que no tenían neumonía dessin.nop \textless-
xyplot(my.nelaal.nop, tr.choice = c(``0 2'', ``0 1''), lwd = 2, layout =
c(1, 2), strip = strip.custom( factor.levels = c(``Sin
neumonía:recuperación'', ``Sin Neumonía: Muertos''), par.strip.text =
list(font = 2)), ylim = c(0, 9), xlim = c(0, 190), xlab = ``Days'',
scales = list(alternating = 1, x = list(at = seq(0, 150, 50)), y =
list(at = seq(0, 8, 2))))

\{r\} \#Plot para los que tenían neumonía dessin.p \textless-
xyplot(my.nelaal.p, tr.choice = c(``0 2'', ``0 1''), lwd = 2, layout =
c(1, 2), strip = strip.custom( factor.levels = c(``Con
neumonía:recuperación'', ``Con neumonía: Muertos''), par.strip.text =
list(font = 2)), ylab = "``, ylim = c(0, 9), xlim = c(0, 190), xlab
=''Days", scales = list(alternating = 1, x = list(at = seq(0, 150, 50)),
y = list(at = seq(0, 8, 2))))

\{r\} \#Ploteamos ambas print(dessin.nop, split = c(1, 1, 2, 1), more =
TRUE, position = c(0, 0, 1.07, 1)) print(dessin.p, split = c(2, 1, 2,
1), position = c(-0.07, 0, 1, 1))

Esto no implica que la neumonía no tenga ningún efecto sobre la
mortalidad. La razón es que la neumonía parece reducir el riesgo de ser
dados de alta Esto implica:

\begin{enumerate}
\def\labelenumi{\arabic{enumi}.}
\item
  \begin{enumerate}
  \def\labelenumii{\arabic{enumii}.}
  \tightlist
  \item
    La neumonía parece reducir el riesgo por todas las causas al final
    de la de la unidad de cuidados intensivos.
  \end{enumerate}
\item
  Los pacientes con neumonía al ingreso permanecen más tiempo en la
  unidad. Durante esta estancia prolongada, están expuestos a un a un
  riesgo de muerte esencialmente
\item
  En consecuencia, mueren más pacientes con neumonía que sin neumonía.
\end{enumerate}

Se trata de un fenómeno típico de riesgos competitivos. Como hay más de
un riesgo que actúan sobre un individuo, no podemos saber, a partir de
un solo riesgo, cuál será el curso futuro de un individuo. Esta
situación se muestra de forma esquemática en la siguiente figura:

\{r, echo=FALSE, out.width = `100\%'\}
knitr::include\_graphics(``pic\_1.PNG'')

Fig. 1: Datos del hospital. Representación esquemática del efecto de la
neumonía con causa específica. El estado de la neumonía no tiene ningún
efecto sobre la causa específica de muerte, que es también un peligro
menor. Un gráfico como el presente podían producirse con el paquete R
compeir; hasta antes de ser deshabilitada del CRAN

Recordemos que una forma de pensar en los peligros de causas específicas
es en términos de fuerzas momentáneas de transición que se mueven a lo
largo de las flechas de los cuadros multiestado. La magnitud de estas
fuerzas se muestra de forma esquemática en la figura 1. La ``fuerza de
muerte'' no está influenciada por el estado de la neumonía, pero la
``fuerza del alta'' se reduce sustancialmente de la neumonía en el
momento del ingreso. La figura 1 ilustra que la ``fuerza global'', es
decir, el riesgo por todas las causas que arrastra un individuo se ve
reducido, conduciendo a una mayor permanencia en la unidad, y que la
fuerza relativa entre las fuerzas específicas de la causa de muerte y de
alta, se ve modificada por el estado de la neumonía.

Notemos que la representación esquemática de la figura 1 tiene
limitaciones. La magnitud de las fuerzas de transición momentánea no
suele ser constante sobre el tiempo, de modo que necesitaríamos toda una
serie de gráficos como los de la figura 1. De hecho, esto se consigue en
la gráfica que ploteamos anteriormente de los estimadores de
Nelson-Aalen; la forma de los estimadores de dichos estimadores, que
estiman los peligros acumulativos, está determinada por los peligros
específicos de la causa. También podemos pensar en la figura 1 de una
manera que no necesariamente ilustre la magnitud de los peligros,
pudiendo variar con el tiempo, sino únicamente los cocientes de los
peligros de muerte y los cocientes de los peligros de descarga,
respectivamente, suponiendose constantes. Este es el enfoque adoptado
por la modelización de los riesgos proporcionales a la causa.

\hypertarget{funciuxf3n-de-incidencia-acumulada}{%
\subsection{Función de incidencia
acumulada}\label{funciuxf3n-de-incidencia-acumulada}}

Por último, comprobamos si nuestra interpretación del análisis de
riesgos acumulativos ha sido correcta observando los estimadores de
Aalen-Johansen de las funciones de incidencia acumulada, nuevamente
estratificadas por el estado de la neumonía. Recordemos que la función
de incidencia acumulada para la muerte, por ejemplo, muestra la
proporción esperada de individuos que mueren en la unidad a lo largo del
tiempo. Si nuestra interpretación del análisis de riesgos acumulativos
ha sido correcta, la función de incidencia acumulativa estimada para la
muerte\ldots\ldots. dentro de los pacientes con neumonía, debería estar
por encima de los pacientes sin neumonía.

Utilizando la función cuminc del paquete cmprsk, calculamos las
estimaciones \(\mathbb{P}( T \leq t, X_{T} = j), j = 1, 2\) (Es decir,
de la función acumulativa de incidencia) dentro de los grupos definidos
\{r\} my.sir.cif \textless- cuminc(my.sir.data\(time, my.sir.data\)to,
group=my.sir.data\$pneu, cencode=``cens'') my.sir.cif

El valor regresado por cuminc es una lista con los componentes ``0 1'',
``1 1'', ``0 2'' y ``1 2''. Los componentes ``0 1'' y ``1 1'' contienen
resultados para el tipo de fallo 1 (Muerte); los componentes ``0 2'' y
``1 2'' contienen resultados para el tipo de fallo 2 (Recuperación). Los
componentes Los componentes ``0 1'' y ``0 2'' son para pacientes con
estado de neumonía 0 al ingreso, es decir, sin neumonía, y los
componentes ``1 1'' y ``1 2'' son para pacientes con estado de neumonía
1.

Esto también lo podemos hacer con la paquetería etm, para matrices de
transiciones; al igual que con mvna. Ejecutamos etm con cada estrato:

\{r\} my.sir.etm.nop \textless- etm(my.sir.data{[}my.sir.data\$pneu ==
0, {]}, c(``0'', ``1'', ``2''), tra, ``cens'', s = 0) my.sir.etm.p
\textless- etm(my.sir.data{[}my.sir.data\$pneu == 1, {]}, c(``0'',
``1'', ``2''), tra, ``cens'', s = 0)

Graficando:

\{r\} op \textless- par(mfrow = c(1, 2)) \#Muerte plot(my.sir.etm.nop,
tr.choice = ``0 1'', conf.int = FALSE, lwd = 2, lty = 1, xlab =
``Días'', ylab = ``Probabilidad'', bty = ``n'', legend = FALSE)
lines(my.sir.etm.p, tr.choice = ``0 1'', conf.int = FALSE, lwd = 2, lty
= 2) legend(0, 0.6, c(``neumonía'', ``sin neumonía''), col = 1, lty =
c(1, 2), bty = ``n'', lwd = 2) title(``Muerte'') axis(1, at = seq(0,
200, 50)) \#\#Dados de alta plot(my.sir.etm.nop, tr.choice = ``0 2'',
conf.int = FALSE, lwd = 2, lty = 1, xlab = ``Días'', ylab =
``Probabilidad'', bty = ``n'', legend = FALSE) lines(my.sir.etm.p,
tr.choice = ``0 2'', conf.int = FALSE, lwd = 2, lty = 2) axis(1, at =
seq(0, 200, 50)) title(``Recuperados'') par(op)

Esta gráfica presenta las estimaciones de Aalen-Johansen
\(\mathbb{P}( b T \leq t, X_{T} = j)\) de las funciones de incidencia
acumulativa para la muerte (izquierda, j = 1) y para el alta (derecha, j
= 2), estratificadas por el estado de la neumonía al ingreso. Las líneas
continuas corresponden a pacientes sin neumonía.

Como era de esperarse, encontramos que mueren más pacientes entre los
que tienen neumonía.

Las estimaciones de Aalen-Johansen \(\mathbb{P}( T \leq t, X_{T} = 1)\)
se muestran en la siguiente gráfica, junto con intervalos de confianza
del 95\% puntuales, generadas por:

\{r\} plot(my.sir.etm.p, tr.choice = `0 1', col = 1, lwd = 2, conf.int =
TRUE, ci.fun = ``cloglog'', legend = FALSE, ylab=``Probability'',
xlim=c(0,190)) lines(my.sir.etm.nop, tr.choice = `0 1', col = ``gray'',
lwd = 2, conf.int = TRUE, ci.fun = ``cloglog'') legend(0, 1,
c(``neumonía'', ``sin neumonía''), col = 1, lty = c(1, 2), bty = ``n'',
lwd = 2) title(``Estimador Aalen-Johansen con intervalos de confianza'')

Los intervalos de confianza apoyan nuestra conclusión anterior de que
finalmente vemos más casos de muerte en el grupo de pacientes con
neumonía. Los gráficos de riesgos acumulados, como en la fgráfica de
riesgos acumulados (La primera gráfica presentada), y los gráficos de
funciones de incidencia acumulada, como en las Figuras 2 y 3, ambos
tienen sus méritos relativos: Obviamente, en la figura 3 es más fácil
saber si la neumonía aumenta la mortalidad unitaria. Sin embargo,
tenemos que examinar los peligros acumulados por causas específicas para
ver si el aumento de la mortalidad se debe a un aumento del peligro de
muerte, o, como en el presente ejemplo, a una disminución del riesgo de
muerte.

\hypertarget{modelo-de-riesgos-proporcionales-de-causa-especuxedfica}{%
\section{Modelo de riesgos proporcionales de causa
específica}\label{modelo-de-riesgos-proporcionales-de-causa-especuxedfica}}

En el apartado anterior dimos un análisis no paramétrico.

Recordemos, en particular, que a partir de las estimaciones de
Nelson-Aalen de los riesgos acumulativos por causas específicas en la
primera gráfica que hicimos, que la neumonía de ingreso aumenta la
mortalidad hospitalaria a través de un efecto decreciente sobre el
riesgo de alta en vida, mientras que el riesgo de muerte en el hospital
se mantiene esencialmente sin cambios. El objetivo del presente análisis
es volver a investigar este hallazgo mediante modelos de riesgos
proporcionales a la causa. Utilizamos el data frame my.sir.data generado
al inicio. La primer gráfica que hicimos sugiere que la neumonía tiene
diferentes efectos en los riesgos específicos de la causa. Por lo tanto,
simplemente ajustamos dos modelos de Cox diferente. Este es el resultado
del análisis de la causa específica de interés para la muerte en el
hospital,

\{r\} fit.pneu.01 \textless- coxph(Surv(time, to == 1) \textasciitilde{}
pneu, my.sir.data) fit.pneu.02 \textless- coxph(Surv(time, to == 2)
\textasciitilde{} pneu, my.sir.data) summary(fit.pneu.02)
summary(fit.pneu.02)

Los análisis de riesgos proporcionales específicos a la causa concuerdan
con nuestros resultados anteriores. El resultado de R anterior también
vuelve a poner de relieve dos aspectos importantes en el análisis de los
datos de riesgos concurrentes: en primer lugar, todos los riesgos
específicos de las causas deben analizarse. No debemos concluir de
ninguna manera a partir de una razón de riesgo de muerte específica por
causa de 0,85 con un intervalo de confianza del 95\% {[}0,503, 1,437{]}
que la neumonía parece no tener impacto en la muerte hospitalaria. En
segundo lugar, los cocientes de riesgo específicos de la causa que se
muestran arriba, no hacen ninguna declaración sobre la magnitud de los
riesgos base específicos de la causa. Esto es muy diferente a nuestro
análisis inicial basado en los estimadores de Nelson-Aalen en la gráfica
inicial.

\{r\} a01.0 \textless- basehaz(fit.pneu.01, centered=FALSE) a02.0
\textless- basehaz(fit.pneu.02, centered=FALSE)

split.screen(figs=c(1,2)) screen(1) plot(c(0, 50), c(0, 5), xlab =
expression(paste(Time, " ``, italic(t))), ylab =''Cumulative
cause-specific hazard``, type =''n``, axes = FALSE, main =''No
neumonía``, cex.main = 1.5, cex.lab = 1.5) axis(1, at=seq(0, 50, 10),
cex.axis=1.5) axis(2, at=seq(0, 5, 1), cex.axis=1.25) box()
lines(a02.0\(time, a02.0\)hazard, type=''s``, lwd=2, lty=2)
lines(a01.0\(time, a01.0\)hazard, type=''s``, lwd=2)
lines(my.nelaal.nop, conf.int = FALSE, col = rep(''darkgray``, 2), lty =
c(1, 2), lwd = 2) legend(0,5,c(''Muerte``,''Recuperado``),
lty=1:2,bty=''n``, cex=1.2, lwd=2) screen(2) plot(x=c(0, 50), y=c(0, 5),
xlab=expression(paste(Time,'' ``, italic(t))), ylab=''Cumulative
cause-specific hazard``, type=''n``, axes=F, main=''Neumonía``,
cex.main=1.5, cex.lab=1.5) axis(1, at=seq(0, 50, 10), cex.axis=1.5)
axis(2, at=seq(0, 5, 1), cex.axis=1.25) box()
lines(a02.0\(time, a02.0\)hazard, type=''s``, lwd=2, lty=2)
lines(a01.0\(time, a01.0\)hazard, type=''s``, lwd=2) lines(my.nelaal.p,
conf.int = FALSE, col = rep(''darkgray``, 2), lty = c(1, 2), lwd = 2)
legend(0,5,c(''Muerte``,''Recuperado``), lty=1:2,bty=''n", cex=1.2,
lwd=2) close.screen(all.screens=TRUE)

La gráfica anterior es de los estimadores de Nelson-Aalen junto con los
estimadores de Breslow (para la ausencia de neumonía) y los estimadores
de riesgo acumulado (para la neumonía al ingreso). Encontramos que todas
las curvas específicas de la muerte en el hospital están en buena
concordancia, al igual que los estimadores de referencia del riesgo
acumulado en el alta. Sin embargo, los estimadores respectivos del
riesgo acumulado en el alta para los pacientes con neumonía no coinciden
tanto, indicando que el efecto de la neumonía en el riesgo en el alta
puede no seguir un modelo de riesgos proporcionales a la causa.

La gráfica sugiere que la razón de riesgo de alta estimada por causa
específica es de 0,336 con un intervalo de confianza del 95\% {[}0,261,
0,434{]} (Es decir, los datos del summary) informa de un efecto
promediado al tiempo de la neumonía sobre el riesgo de alta.

\hypertarget{modelo-de-subdistribuciuxf3n-de-riesgos-proporcionales}{%
\section{Modelo de subdistribución de riesgos
proporcionales}\label{modelo-de-subdistribuciuxf3n-de-riesgos-proporcionales}}

Consideramos el modelo de subdistribución

\{r\} fit.sir\textless-crr(ftime =
my.sir.data\(time, fstatus = my.sir.data\)to,cov1 = my.sir.data\$pneu,
failcode = ``1'', cencode = ``cens'') fit.sir

El valor reportado, es el p-value correspondiente a un test del tipo
log-rank para la subdistribución de riesgos (Abarcaremos esto más a
detalle al final, en la conclusión)

Veamos un summary: \{r\} summary(fit.sir)

Vemos que el coeficiente es significativo: Encontramos un efecto
significativo de la neumonía en la función de incidencia acumulada por
muerte hospitalaria 2.65 veces mayor (usando el estimador puntual),
puesto que nos da un intervalo de confianza para el coeficiente de
\([1.63,4.32]\); como el coeficiente es positivo, en el contexto de
riesgos competitivos, nos indica que la neumonía incrementa la
mortalidad en los hospitales (Es decir, de las personas que se
encuentran ya hospitalizadas), sin embargo, no muestra algún efecto en
la causa específica para muerte de pacientes hospitalizados. El
incremento en la mortalidad fue debido a un considerable decrecimiento
del riesgo de causa específica para vivos dados de alta. (Todo esto
porque la base es que no ingresó con neumonía)

Ahora, usemos dicho modelo para modelar la CIF:

\{r\} daddeln \textless- predict.crr(fit.sir, cov1 = matrix(c(0, 1),
nrow = 2))

\hypertarget{predicciuxf3n-sin-neumonuxeda}{%
\subsection{predicción sin
neumonía}\label{predicciuxf3n-sin-neumonuxeda}}

split.screen(figs=c(1,2)) screen(1) plot(c(0, 25), c(0, 0.2), xlab =
expression(paste(Tiempo, " ``, italic(t))), ylab =''``, type =''n``,
axes=FALSE, main =''Sin neumonía``, cex.main = 1.5, cex.lab = 1.5)
axis(1, at = seq(0, 25, 5), cex.axis = 1.5) axis(2, at = seq(0, 0.2,
0.05), cex.axis = 1.25) box() mtext(text=''CIF por estado de neumonía``,
side = 2, line = 3, cex = 1.25)
lines(my.sir.cif\(`0 1`\)time,my.sir.cif\(`0 1`\)est, type =''s``, lwd =
2,lty = 1) \#\# predicción con neuonía lines(daddeln{[},1{]},
daddeln{[},2{]}, type =''s``, lwd = 2, lty = 1, col =''darkgrey``)
screen(2) plot(c(0, 25), c(0, 0.2), xlab = expression(paste(Tiempo,''
``, italic(t))), ylab =''``, type =''n``, axes = FALSE, main =''Con
neumonía``, cex.main = 1.5, cex.lab = 1.5) axis(1, at = seq(0, 25, 5),
cex.axis = 1.5) axis(2, at = seq(0, 0.2, 0.05), cex.axis = 1.25) box()
mtext(text=''CIF por estado de neumonía``, side = 2, line = 3, cex =
1.25) lines(my.sir.cif\(`1 1`\)time,my.sir.cif\(`1 1`\)est, type =''s``,
lwd = 2,lty = 1) \#\# predicted curves lines(daddeln{[},1{]},
daddeln{[},3{]}, type =''s``, lwd = 2, col =''darkgrey")
close.screen(all.screens=TRUE)

Las líneas negras están dadas por el estimador de Aalen-Johansen, y las
grises por el modelo bajo el supuesto de modelo de subdistribución de
riesgos proporcionales; vemos para las personas sin neumonía es
prácticamente la misma, pero no es el caso para las personas con
neumonía inicialmente.

\hypertarget{entonces-suxed-influye-el-que-ingrese-con-neumonuxeda-o-sin-neumonuxeda}{%
\section{Entonces, ¿Sí influye el que ingrese con neumonía o sin
neumonía?}\label{entonces-suxed-influye-el-que-ingrese-con-neumonuxeda-o-sin-neumonuxeda}}

Revisamos brevemente el ejemplo de datos hospitalarios, donde el
objetivo era investigar el impacto de neumonía diagnosticada al ingreso.
El resultado no fue una diferencia significativa entre los peligros
específicos de la causa de muerte en el hospital, pero los peligros
específicos de la causa significativamente diferentes para el alta con
vida. Este ejemplo, se ilustran las limitaciones de probar la igualdad
de riesgos de causa específica. Nuestro análisis anterior mostró que la
neumonía aumentó el número de pacientes que mueren en el hospital,
aunque el riesgo para muerte hospitalaria se encontrado que es similar
con o sin neumonía.

El hallazgo encontrado en nuestro análisis ha sido que una
interpretación adecuada de un análisis de riesgos competitivos requiere
una consideración cuidadosa de todos los riesgos específicos de la
causa, incluidos los signos y las magnitudes de los efectos de, por
ejemplo, el estado de neumonía así como la magnitud relativa de los
riesgos específicos de causa única dentro de un grupo.

El modelo anterior se usa a menudo para estudiar directamente el impacto
de covariables como la neumonía, Este modelo es un modelo de tipo Cox
para el riesgo de subdistribución, y que la subdistribución el peligro
restablece una correspondencia biyectiva con la función de incidencia
acumulada de interés. Por lo tanto, el p-value obtenido al ajustar tal
modelo debe ser adecuado para comparar funciones de incidencia acumulada
directamente. Ilustramos esto brevemente con los datos del hospital.

\{r\} fit.sir

Encontramos un efecto significativo de la neumonía en la función de
incidencia acumulada por muerte hospitalaria. Esta prueba se debe a Gray
y, por lo tanto, a menudo se llama Prueba de Gray, y consiste en un test
no paramétrico en el que se comparan dos o más CIFs. El test es análogo
al log-rank para comparar curvas derivadas de los estimadores de
Kaplan-Meier, usando una estadística de prueba modíficada de la
ji-cuadrada. La prueba puede derivarse en analogía con nuestra
derivación de la prueba log-rank ya vista en clase, si usamos los
incrementos del estimador de la función de riesgo de subdistribución
acumulativa.

\hypertarget{comprobaciuxf3n-de-supuestos}{%
\section{Comprobación de supuestos}\label{comprobaciuxf3n-de-supuestos}}

Se necesita, en riesgos proporcionales en la CIF.

Para comprobar la hipótesis de proporcionalidad de la regresión de
riesgos competitivos, podemos plotear \(ln(-ln(1-F))\) contra
\(ln(tiempo)\), donde F es la CIF del evento de interés.

\{r\} fit=cuminc(my.sir.data\(time,my.sir.data\)to,cencode =
``cens'',strata=my.sir.data\(pneu) a=timepoints(fit,times=my.sir.data\)time)
cif=t(a\(est[1:2,]) llcif=log(-log(1-cif)) matplot(log(unique(sort(my.sir.data\)time))),llcif,pch=c(1,3),
col=1, xlab=`Tiempo para la muerte', ylab=`log(-log(1-CIF))')

Notamos que las curvas difieren a una distancia casi similar durante
todo el tiempo, lo que prueba que dichos riesgos son proporcionales

En el modelado deriesgos de causa específia, se asume que el logaritmo
del riesgo cambia linealmente con la covariable. Puede ser revisada
categorizando la covariable y examinando los efectos para cada
categoría; sin embargo nuestras variables de interés ya son categóricas,
y el test de Gray nos garantiza que hay diferencia entre estas; por lo
que se cumple la linealidad.

\end{document}
